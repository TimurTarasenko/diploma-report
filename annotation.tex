\section*{Аннотация}
\thispagestyle{empty}

\begin{center}
  \begin{minipage}{0.8\textwidth}
    на дипломный проект ``Методы распознавания, захвата и сопровождения движущихся объектов и их применение в задаче отслеживания людей'' студента УО ``Белорусский государственный университет информатики и радиоэлектроники'' Чернецкого И.Н.
  \end{minipage}
\end{center}

Дипломный проект выполнен на 6 листах формата А1 с пояснительной запиской на 67 страницах (без приложений справочного или информационного характера). Пояснительная записка включает 6 глав, 18 рисунков, 22 литературных источников.

Темой дипломной работы является актуальная проблема компьютерного зрения --- захват людей на изображении и оценка их позы. Целью данного дипломного проекта является разработка приложения для захвата людей на изображении. ПО может быть внедрено в другие проекты с целью получения информации об объекте на изображении.

В первой главе дипломного проекта дается обзор оператора обнаружения краев Канни.

Во второй главе рассматривается метод усиления простых классификаторов, приводится алгоритм AdaBoost и его математическое обоснование.

Третья глава посвящена описанию дескриптора признаков Shape Context и изложению необходимого математического аппарата. Также излагаются возможные области применения.

В четвертой главе рассмотрены графические структуры, приведен метод захвата людей на изображении, совершен синтез изложенного в предыдущих разделах в единый фреймворк.

Пятая глава посвящена реализации пространственно\hyp{}антропометрической эргономической совместимости работника и технического средства при организации рабочего места.

В шестой главе приводится расчет себестоимости разработки, дается расчет экономического эффекта от использования программного средства. 

Глава ``Заключение'' содержит краткие выводы по дипломному проекту.

\newpage
