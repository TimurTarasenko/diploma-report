\section*{ВВЕДЕНИЕ}
\addcontentsline{toc}{section}{Введение}

В последние время, благодаря интенсивному развитию информационных технологий, процесс снятия изображения и/или видео и его хранение достаточны просты. Более того, на сегодняшний момент весьма недороги. Поэтому все большее распространение получают системы наружного наблюдения, системы индексации видео, системы автомобильной безопасности и т.д.

Целью данного дипломного проекта являются описание и разработка обобщенного фреймворка для захвата людей на изображении и оценки их позы, основанного на графических структурах (pictorial structures).

В современном мире обширнейшие электронные коллекции изображений и даже видео --- не редкость. Оцифровка и хранение больших объемов визуальных материалов не является проблемой с технической точки зрения.

До недавнего времени традиционным считался поиск визуальной информации, опирающийся на индексирование текстовых описаний, ассоциированных с изображением или видео.

При очевидной необходимости организации доступа к визуальной коллекции посредством поиска по текстовой информации, такой подход представляется недостаточным. Неоднозначность соответствия между визуальным содержанием и текстовым описанием снижает показатели точности и полноты поиска.

В связи с этим возникает проблема организации доступа к современным электронным коллекциям изображений и видео и использованием комплекса средств --- как текстовых описаний, так и характеристик визуального содержания, простейших типа цветовой гаммы и более сложных, связанных с распознаванием образов. Текстовое описание и визуальная поисковая информации дополняют друг друга, обеспечивая возможность разностороннего поиска. Часть поисковой информации предоставляют выходные данные программного обеспечения, разработанного в рамках данного дипломного проекта. Используя ее, можно разработать поисковую систему для систем наружного наблюдения, которая позволит выполнять запросы вида: ``показать кадры определенного участка, на которых присутствует хотя бы один человек в заданное время''.

Другой областью применения результатов проекта являются системы видеонаблюдения для обеспечения безопасности и охраны. Такие системы способны определить, где находится субъект, в каком он направлении движется и оценить в какой позе он находится. Это позволяет значително уменьшить нагрузку с оператора данной системы, так как в этом случае ему не надо следить постоянно за всеми камерами одновременно. Ведь с помощью автоматического захвата людей на видео, поступающего с камер наблюдения, например, можно реализовать функцию системы, которая будем сигнализировать в том случае, если на обозреваемом участке появился какой-либо субъект.

Также можно реализовать функцию поиска определенных действий на видео, например, поиск всех кадров, на которых присутствует человек с поднятыми руками.

До недавнего времени методы захвата людей не могли быть применены в широком спектре задач из-за отсутствия эффективных универсальных алгоритмов и подходов. Здесь мы пытаемся продемонстрировать последние достижения этого раздела знаний в области компьютерного зрения. В \cite{andriluka09} показано на эксперементах, что этот подход превосходит другие методы на трех наборах тестовых данных. Следует особо подчеркнуть, что другие методы разработаны с учетом специфики входных данных. Это наглядно демонстрирует такое свойство демонстрируемого метода, как обобщенность.

Несмотря на это, предоставленная модель на удивление проста. Это стало возможным, по нашему мнению, благодаря двум ключевым компонентам:
\begin{itemize}
  \item строго дифференцирующей модели представления, которая позволяет надежно захватывать отдельные части объекта на изображении,
  \item и моделированию конфигурации объекта с помощью графических структур.
\end{itemize}

В данном дипломном проекте рассматривается современный метод захвата людей на изображении и оценки их позы, основанный на графических структурах, а также производится его реализация.

\newpage
